\documentclass{article}
\usepackage{geometry}
\usepackage{fancyhdr}
\usepackage{amsmath,amsthm,amssymb}
\usepackage{graphicx}
\usepackage{hyperref}
\usepackage{color}
\usepackage{listings}
\usepackage{verbatim}

\lstset{ %
language=C,                % choose the language of the code
basicstyle=\footnotesize,       % the size of the fonts that are used for the code
numbers=left,                   % where to put the line-numbers
numberstyle=\footnotesize,      % the size of the fonts that are used for the line-numbers
stepnumber=1,                   % the step between two line-numbers. If it is 1 each line will be numbered
numbersep=5pt,                  % how far the line-numbers are from the code
backgroundcolor=\color{white},  % choose the background color. You must add \usepackage{color}
showspaces=false,               % show spaces adding particular underscores
showstringspaces=false,         % underline spaces within strings
showtabs=false,                 % show tabs within strings adding particular underscores
frame=single,                   % adds a frame around the code
tabsize=2,              % sets default tabsize to 2 spaces
captionpos=b,                   % sets the caption-position to bottom
breaklines=true,        % sets automatic line breaking
breakatwhitespace=false,    % sets if automatic breaks should only happen at whitespace
escapeinside={\%}{)}          % if you want to add a comment within your code
}
\hypersetup{pdfborder={0 0 0}} % die, red boxes!
\begin{document}
Abstract: in this paper, I propose algorithms to compute the result of 
bitwise interval arithmetic for several integer operators common to the 
C family of programming languages.  I will also provide efficient 
implementations for each, and will attempt to demonstrate the correctness 
of both the algorithms and the implementations.

\section{Definitions}

In low level programming, integers are represented by sequences of $N$ 
bits. Let us refer to the set of N-bit bit sequences as $B_N$. 
For this discussion, we will assume the first bit (low bit) 
is indexed at $0$, and 
the high bit at $N-1$. We will be abusing interval syntax:
\begin{equation*} 
[a,b] = \{i \in \mathbb{Z} : a \le i \le b\}
\end{equation*}
For convenience, let us define
\begin{equation*}
I_N = [0, N)
\end{equation*}
\begin{equation*}
I_0 = \emptyset
\end{equation*}
A bit sequence $a \in B_N$ may be interpreted as a signed or unsigned 
integer, with the semantics as follows:
\begin{equation}\label{unsf} 
unsigned : B_N \rightarrow [0,2^N)
\end{equation}
\begin{equation}\label{sgnf} 
signed : B_N \rightarrow [-2^{N-1},2^{N-1})
\end{equation}
\begin{equation}\label{unsdef} 
unsigned(a) = \sum_{i\in I_N} {a_i} \cdot 2^i
\end{equation}
\begin{equation}\label{sgndef} 
signed(a) = - a_{N-1} \cdot 2^{N-1} + \sum_{i = 0}^{N-2} {a_i} \cdot 2^i
\end{equation}

It should be apparent that $B_N$ is isomorphic to $\mathbb{Z}_N$. 
Hence, we may use the modular operators +,-,*, on elements of $B_N$, with 
the result in the range of signed or unsigned, depending on context.
Aaaaa! C standard specifies that result of arithmetic operations on 
signed integers
which causes overflow is undefined!!!!!

Also, we define $a>b$ as 

\begin{equation*}t(a)>t(b), \quad for \quad a,b \in B_N\end{equation*}

for $t$ $signed$ or $unsigned$, depending on context.
\vspace{10pt}

The operators of interest to us are functions of the form 
$f : B_N \times B_N \rightarrow B_N$. 
They may be computed in $O(N)$ time. In particular, on modern 
computers, they may be executed very efficiently for select values of $N$.
We will refer to them by the symbols used in C. They are the 
following:
\begin{itemize}
\item \& - bitwise and, where

$(a \& b)_i = a_i \wedge b_i$, for $i \in I_N$
\item \textbar \,  - bitwise or, where

$(a$ \textbar $b)_i = a_i \vee b_i$, for $i \in I_N$
\item \verb!^!  - bitwise exclusive or, where

$(a \verb!^! b)_i = a_i \Leftrightarrow \neg b_i$, for $i \in I_N$
\end{itemize}

Other C operators which will be used:

\begin{itemize}
\item $\sim$ - bitwise negation, 

$ (\sim) : B_N \rightarrow B_N$, where

$( \sim a)_i = \neg a_i$, for $i \in I_N$

It should be apparent that $(\sim)$ is a bijection, and that $(\sim)$ is its 
own inverse.
\item \textless\textless \,  - left shift

$ ($\textless\textless$) : B_N \times I_N  \rightarrow B_N$, where
\begin{equation*} 
(a \text{\textless\textless} \, b)_i = \left\{
    \begin{array}{l l}
    a_{i-b} & \quad \text{if $i \ge b$}\\
    0 & \quad \text{otherwise}
    \end{array}\right.
\end{equation*}
\item \textgreater\textgreater \,  - right shift

$ ($\textgreater\textgreater$) : B_N \times I_N  \rightarrow B_N$, where
\begin{equation*} 
(a \text{\textgreater\textgreater} \, b)_i = \left\{
    \begin{array}{l l}
    a_{i+b} & \quad \text{if $i+b < N$}\\
    0 & \quad \text{otherwise}
    \end{array}\right.
\end{equation*}
\item bsr

This is not a C primitive
(though it is a primitive in Intel assembly code), 
but it is used extensively in this paper.
It takes the name of an Intel instruction. Its semantics are as 
follows:

$bsr : B_N \setminus \{\vec 0\} \rightarrow I_N$

$bsr(a) = max \,\{ i \in I_N : a_i = 1\} $

Otherwise, $bsr$ can be implemented in user code.
\end{itemize}
Todo: check C standard for gotchas
\vspace{10pt}

Define functions
\vspace{10pt}

$up : B_N \times I_N \rightarrow B_N,$

$dn : B_N \times I_N \rightarrow B_N$

$up(a,i) = \alpha$, where
\begin{equation*}
    \alpha_j = \left\{
    \begin{array}{l l}
    a_j & \quad \text{if $j > i$}\\
    1 & \quad \text{if $j = i$}\\
    0 & \quad \text{if $j < i$}\\
    \end{array}\right.
\end{equation*}

$dn(a,i) = \alpha$, where
\begin{equation*}
    \alpha_j = \left\{
    \begin{array}{l l}
    a_j & \quad \text{if $j > i$}\\
    0 & \quad \text{if $j = i$}\\
    1 & \quad \text{if $j < i$}\\
    \end{array}\right.
\end{equation*}

Note that with unsigned interpretation, and 
for $x = up(a,i)$, where $a_i = 0$, and $y \in B_N$ the following hold:
\begin{equation*}
a \le x
\end{equation*}
\begin{equation*}
a \le y \wedge y_i = 1 \quad \Rightarrow \quad x \le y
\end{equation*}

For $x = dn(a,i)$, where $a_i = 1$, and $y \in B_N$ the following hold:
\begin{equation*}
a \ge x
\end{equation*}
\begin{equation*}
a \ge y \wedge y_i = 0 \quad \Rightarrow \quad x \ge y
\end{equation*}

Let $s(a,i)$ be the bits $a_{0}$ through $a_{i}$ inclusive.

Given integer intervals 
\begin{equation*}
a = [a_{min},a_{max}] \subseteq B_N
\end{equation*} 
\begin{equation*}
b = [b_{min},b_{max}] \subseteq B_N
\end{equation*} 
let
\begin{equation*}
c = [c_{min},c_{max}] \subseteq B_N
\end{equation*} 
where
\begin{equation*}c_{min} = min \; \{A \; op \; B :A \in a, B \in b\},\end{equation*} 
\begin{equation*}c_{max} = max \; \{A \; op \; B :A \in a, B \in b\}, \end{equation*}
for bitwise operator $op$, and $signed$ or $unsigned$ interpretation.

Also, let $\check a, \hat a \in a$, and likewise for $b$, where
\begin{equation*}
c_{max} = \hat a \; op \; \hat b
\end{equation*}
\begin{equation*}
c_{min} = \check a \; op \; \check b
\end{equation*}


\section{unsigned $max \{a\&b\}$}\label{maxand}
Consider $i \in I_N$ and ranges $a^i$ and $b^i$, where

\begin{equation}\label{le1} s(a^i_{min},i) \le s(a^i_{max},i)\end{equation}
and likewise for $b$. Then by \eqref{unsdef} exactly one of the following must be true:
\begin{equation}\label{00} 
a^i_{min,i},a^i_{max,i} = 0,0
\end{equation}
\begin{equation}\label{01} 
a^i_{min,i},a^i_{max,i} = 0,1
\end{equation}
\begin{equation}\label{11} 
a^i_{min,i}, a^i_{max,i} = 1,1
\end{equation}
note that:
\begin{equation}\label{up1}
a^i_{max, i} = 1 \quad \Rightarrow \quad \alpha_i = 1 \quad 
\text{for some} \; \alpha \in a^i
\end{equation}
and likewise for $b$.

It follows from \eqref{unsdef} that for $\alpha \in a^i$, 
$\beta \in b^i$:
\begin{equation}
(\alpha \; \text{\&} \; \beta)_i = 1
\quad \Rightarrow \quad c_{max,i} = 1
\end{equation}
then from \eqref{up1}, it follows that 
\begin{equation}
c_{max,i} = a^i_{max,i} \; \text{\&}\; b^i_{max,i}
\end{equation}

Now, for $i \ge 1$, in the cases \eqref{00} and \eqref{11}, 
it follows that \eqref{le1} 
will hold for $a^{i-1}=a^i$ and $b^{i-1} = b^i$, but possibly
not for case \eqref{01}. Note that:

\begin{equation*}
\eqref{01} \quad \Rightarrow 
\quad (c_{max,i} = 1 \quad \Leftrightarrow \quad 
b^i_{max,i} = 1)
\end{equation*}


For $\eqref{01} \wedge b^i_{max,i} = 1$, it must be that $\hat a_i = 1$. 
Then for
\begin{equation}\label{011maxand} 
a^{i-1} \quad = \quad [up(a^i_{min}, i), a^i_{max}]
\end{equation}

$\hat a \in a^{i-1}$, and
\eqref{le1} holds because $a^{i-1}_{min,j} = 0$ for $j < i$.
Likewise for $b$.
\vspace{10pt}

For $\eqref{01} \wedge b^i_{max,i} = 0$, $\hat a_i$ is unconstrained.
Since $c_{max}$ will be maximized with $a^j_{max,j} = 1$, 
$\hat a_i = 0$ is desirable. Then for
\begin{equation}\label{010maxand} 
a^{i-1} \quad = \quad [a^i_{min}, dn(a^i_{max},i)]
\end{equation}

$\hat a \in a^{i-1}$, and
\eqref{le1} holds because $\bar a_{i-1,max,j} = 1$ for $j<i$.  
Likewise for $b$.
\vspace{10pt}

So, given ranges $a$ and $b$ for which to find $c_{max}$ for $a\&b$,
\eqref{le1} is true for $\bar a^{N-1}=a$, $\bar b^{N-1}=b$, and by
\eqref{010maxand} and \eqref{011maxand}, \eqref{le1} is true for 
subsequent $\bar a_i$, $\bar b_i$. Then using this procedure 
we can compute the complete value of $c_{max}$. Prove it. No.
\vspace{10pt}

An intrepid observer will note that only the first applications of
\eqref{011maxand} and \eqref{010maxand} affect subsequent calculations, and only
applications of \eqref{010maxand} will affect the value of $c_{max}$. 
However, to find the first application of each without computing all of $a^i$
and $b^i$, one must consider the dependencies they
may have on each other. Call a dependency chain a sequence of applications of 
\eqref{011maxand} and \eqref{010maxand} on $a$ or $b$
\begin{equation*}
d_1 \rightarrow d_2 \rightarrow ... \; d_n
\end{equation*}
where $d_i$ occurs at bit $j_i$, $j_i < j_{i+1}$, and there is at most 1 
application of each type on each of $a$ and $b$.
\vspace{10pt}

\begin{tabular}{l|c|c|c|c|l r}
& $a_{min,i}$ & $a_{max,i}$ & $b_{min,i}$ & $b_{max,i}$ & effect &\\
\hline
$\eqref{011maxand}_a$ & $0$ & $1$ & & $1$ & $a_{min,j} := 0$, \;& $j < i$ \\
\hline
$\eqref{011maxand}_b$ & & $1$ & $0$ & $1$ & $b_{min,j} := 0$, \;& $j < i$ \\
\hline
$\eqref{010maxand}_a$ & $0$ & $1$ & & $0$ & $a_{max,j} := 1$, \;& $j < i$ \\
\hline
$\eqref{010maxand}_b$ & & $0$ & $0$ & $1$ & $b_{max,j} := 1$, \;& $j < i$ \\
\end{tabular}

\vspace{10pt}

Then\\
$\eqref{011maxand}_a$ \quad can trigger \quad $\eqref{010maxand}_a$\\
$\eqref{011maxand}_b$ \quad can trigger \quad $\eqref{010maxand}_b$\\
$\eqref{010maxand}_a$ \quad can trigger \quad $\eqref{011maxand}_a$, $\eqref{011maxand}_b$\\
$\eqref{010maxand}_b$ \quad can trigger \quad $\eqref{011maxand}_a$, $\eqref{011maxand}_b$\\

Because $\eqref{010maxand}_b$ sets $b_{max} = 1$ for following bits
and $\eqref{010maxand}_a$ at $i$ requires $b_{max,i} = 0$ and vice versa,
$\eqref{010maxand}_a$ and $\eqref{010maxand}_b$ cannot coexist in a 
dependency chain, and if $\eqref{010maxand}_a$ occurs, $\eqref{010maxand}_b$ 
will not.
Then the worst possible dependency
for the first application of $\eqref{010maxand}_a$ is:
\begin{equation*} 
\eqref{011maxand}_a \rightarrow \eqref{010maxand}_a
\end{equation*}

and likewise for $b$. 

Let us introduce the syntax $aabb_i$ to mean 
$a_{min,i},a_{max,i},b_{min,i},b_{max,i}$.

Then the algorithm to find $max \{a\&b\}$ can be expressed as follows:
\vspace{10pt}

Given ranges $a,b$, define $i_a$ as the highest bit where 
$\eqref{010maxand}_a$ occurs, and likewise for $b$. Then

\begin{equation*}
i_a = max \, ( \, \{i_0\} \cup \{i \in I_{i_1}\; : \; 
            aabb_i = *1*0\} \, ) 
\end{equation*}

where
\begin{equation*}
i_0 = max \, \{i \in I_N\; : \; 
aabb_i = 01*0\},\,
\text{or $-1$ if the set is empty}
\end{equation*}
\begin{equation*}
i_1 = max \, \{i\in I_N\; : \; 
aabb_i = 01*1\},\,
\text{or $-1$ if the set is empty}
\end{equation*}
\vspace{0pt}

if $i_a > i_b$, then

\quad $c_{max} = dn\,(a_{max},i_a) \; \text{\&} \; b_{max} $

otherwise if $i_b > -1$,

\quad $c_{max} = a_{max} \; \text{\&} \; dn\,(b_{max},i_b) $

otherwise,

\quad $c_{max} = a_{max} \; \text{\&} \; b_{max} $
\vspace{10pt}

This algorithm is correct because I said so. Sure would like to have 
better justification than that.
\vspace{10pt}

Given a suitable N, (we will assume $N=32$),
$dn(a_{max},i_a)$ can be efficiently computed with the C 
primitives listed above

Then in C, given ranges $a$ and $b$:
\begin{lstlisting}
#include<stdint.h>
uint32_t i_a(uint32_t amin, uint32_t amax, uint32_t bmax){
\end{lstlisting}
We can compute $i_a$ as follows:

\begin{lstlisting}
    uint32_t xi0 = ~amin & amax & ~bmax;
\end{lstlisting}
to find the bits unmodified by \eqref{011maxand} which trigger \eqref{010maxand}. 
When $xi0 \ne 0$, $i_0 = bsr(xi0)$. Otherwise, $i_0 = -1$.
\begin{lstlisting}
    uint32_t xi1 = ~amin & amax & bmax;
\end{lstlisting}
to find the bits which trigger \eqref{011maxand}. 
Assuming $xi1 \ne 0$, $i_1 = bsr(xi1)$.
\begin{lstlisting}
    uint32_t xi01 = 0;
    if (xi1 != 0) xi01 = ((1 << bsr(xi1))-1) & amax & ~bmax;
\end{lstlisting}
to find the bits modified by \eqref{011maxand} which trigger \eqref{010maxand}.
Note that
\begin{lstlisting}
    ((1 << i)-1) 
\end{lstlisting}
will result in a bit sequence with all bits below bit $i$ set to $1$, and 
all other bits set to $0$. In particular, $i=0$ will result in $0$. 

Assuming $xi01 \ne 0$, $i_a = max(i_0, bsr(xi01))$. Without using $max$,
this can be computed as follows:
\begin{lstlisting}
    uint32_t xa = xi0 | xi01;
    uint32_t ia = -1;
    if (xa != 0) ia = bsr(xa);
    return ia;
}
\end{lstlisting}

Then, to compute $dn(a_{max}, i_a)$:
\begin{lstlisting}
uint32_t dn(uint32_t amin, uint32_t amax, uint32_t bmax){
    uint32_t mask = 0;
    uint32_t ia = i_a(amin,amax,bmax);
    if (ia != -1) mask = (1 << ia)-1;
    return amax|mask;
}
\end{lstlisting}

Note that setting $a_{max,i} = 0$ has no effect on $c_{max}$, so it is not
performed.
Also, since $a\text{\textbar}0 = a$, the case $ia = -1$ has the same effect
as $ia = 0$. Thus, if
\begin{equation}\label{oho}
min(ia) = 0
\end{equation} 
we obviate the
need for the conditional.
We can also use this principle to remove conditionals which guard 
$bsr$ by setting its argument's minimum value to $1$. 
$arg \text{\textbar} 1$ can accomplish this.
With these optimizations, our code is now:
\begin{lstlisting}
#include<stdint.h>
uint32_t i_a(uint32_t amin, uint32_t amax, uint32_t bmax){
    uint32_t xi0 = ~amin & amax & ~bmax;
    uint32_t xi1 = ~amin & amax & bmax;
    uint32_t xi01 = ((1 << bsr(xi1|1))-1) & amax & ~bmax;
    uint32_t xa = xi0 | xi01;
    uint32_t ia = bsr(xa|1);
    return ia;
}

uint32_t dn(uint32_t amin, uint32_t amax, uint32_t bmax){
    uint32_t mask = (1 << i_a(amin,amax,bmax))-1;
    return amax|mask;
}
\end{lstlisting}

Altogether, the algorithm expressed in C is as follows:
\begin{lstlisting}
#include <stdint.h>
uint32_t maxand(uint32_t amin, uint32_t amax, uint32_t bmin, uint32_t bmax){
    uint32_t xa1 = ~amin& amax      & bmax;
    uint32_t xa0 = ~amin& amax      &~bmax;
    uint32_t xb1 =        amax&~bmin& bmax;
    uint32_t xb0 =       ~amax&~bmin& bmax;
    xa0 |= ((1 << bsr(xa1|1))-1)&amax&~bmax;
    xb0 |= ((1 << bsr(xb1|1))-1)&bmax&~amax;
    if(xa0 > xb0) amax |= (1 << bsr(xa0|1))-1;
    else  bmax |= (1 << bsr(xb0|1))-1;
    return amax&bmax;
}
\end{lstlisting}

\section{unsigned $max\{a\text{\textbar} b\}$}
Again, the same basic construction as \ref{maxand} may be used. 
The interesting differences are:

It follows from \eqref{unsdef} that for $\alpha \in a^i$, $\beta \in b^i$:
\begin{equation}
(\alpha \; \text{\textbar} \; \beta)_i = 1
\quad \Rightarrow \quad c_{min,i} = 1
\end{equation}

then from \eqref{up1}, it follows that 
\begin{equation}\label{maxoreq}
c_{max,i} = a^i_{max,i} \; \text{\textbar}\; b^i_{max,i}
\end{equation}

Then
\begin{equation*}
\eqref{01} \quad \Rightarrow \quad c_{max,i} = 1
\end{equation*}

For $\eqref{01} \wedge b^i_{max,i} = 0$, then $ \hat a_i = 1$ must be true. 
Then 
\begin{equation}\label{011maxor} 
a^{i-1} \quad = \quad [up(a^i_{min}, i), a^i_{max}]
\end{equation}

Likewise for $b$.
\vspace{10pt}

For $\eqref{01} \wedge b_{max,i} = 1$, $\hat a_i$ is unconstrained.
Since $c_{max}$ is maximized when $a_{max,j} = 1$, 
$\hat a_i = 0$ is desirable. Then
\begin{equation}\label{010maxor} 
a^{i-1} \quad = \quad [a^i_{min}, dn(a^i_{max},i)]
\end{equation}

Likewise for $b$.
\vspace{10pt}

Now it is \eqref{010maxor} that affects the result, therefore the first
bit where it occurs must be found.
\vspace{10pt}

\begin{tabular}{l|c|c|c|c|l r}
& $a_{min,i}$ & $a_{max,i}$ & $b_{min,i}$ & $b_{max,i}$ & effect &\\
\hline
$\eqref{011maxor}_a$ & $0$ & $1$ & & $0$ & $a_{min,j} := 0$, \;& $j < i$ \\
\hline
$\eqref{011maxor}_b$ & & $0$ & $0$ & $1$ & $b_{min,j} := 0$, \;& $j < i$ \\
\hline
$\eqref{010maxor}_a$ & $0$ & $1$ & & $1$ & $a_{max,j} := 1$, \;& $j < i$ \\
\hline
$\eqref{010maxor}_b$ & & $1$ & $0$ & $1$ & $b_{max,j} := 1$, \;& $j < i$ \\
\end{tabular}
\vspace{10pt}

Then\\
$\eqref{011maxor}_a$ \quad can trigger \quad $\eqref{010maxor}_a$\\
$\eqref{011maxor}_b$ \quad can trigger \quad $\eqref{010maxor}_b$\\
$\eqref{010maxor}_a$ \quad can trigger \quad $\eqref{011maxor}_a$,
$\eqref{010maxor}_b$ \\
$\eqref{010maxor}_b$ \quad can trigger \quad $\eqref{011maxor}_b$,
$\eqref{010maxor}_a$ \\

Note that by \eqref{maxoreq}, the effect of $\eqref{010maxor}_b$ 
occurring at bit $i$ is that $c_{max,j} = 1$, for $j<i$. 
This makes any
application of $\eqref{010maxor}_a$ occurring at bit $j<i$ superfluous.
Therefore, we need only find the first of $\eqref{010maxor}_a$ and 
$\eqref{010maxor}_b$.
Thus, we need only take 
\begin{equation*} 
\eqref{011maxor}_a \rightarrow \eqref{010maxor}_a
\end{equation*}
into account, and likewise for $b$.
Therefore, the algorithm for max\{a\text{\textbar}b\} is:
\vspace{10pt}

Given ranges $a$ and $b$, define $i_a$ as the first bit where $\eqref{010maxor}_a$ occurs.

\begin{equation*}
i_a = 
max \, ( \, \{i_1\} \cup 
\{ i \in I_{i_0} \; : \; aabb_i = *1*1\} )
\end{equation*}

    where
\begin{equation*}
i_0 = max \, \{i \in I_N\; : \; 
aabb_i = 01*0\},\,
\text{or $-1$ if the set is empty}
\end{equation*}
\begin{equation*}
i_1 = max \, \{i\in I_N\; : \; 
aabb_i = 01*1\},\,
\text{or $-1$ if the set is empty}
\end{equation*}
\vspace{0pt}

if $i_a > i_b$, then

\quad $c_{max} = dn\,(a_{max},i_a) \; \text{\textbar} \; b_{max} $

otherwise if $i_b > -1$,

\quad $c_{max} = a_{min} \; \text{\textbar} \; dn\,(b_{max},i_b) $

otherwise,

\quad $c_{max} = a_{max} \; \text{\textbar} \; b_{max} $
\vspace{10pt}

Then in C, given ranges $a$ and $b$,
We can compute $i_a$ as follows:
\begin{lstlisting}
#include<stdint.h>
uint32_t i_a(uint32_t amin, uint32_t amax, uint32_t bmax){
    uint32_t xi0 = ~amin & amax & ~bmax;
\end{lstlisting}
to find the bits where \eqref{011maxor} occurs.
Assuming $xi0 \ne 0$, $i_0 = bsr(xi0)$.
\begin{lstlisting}
    uint32_t xi1 = ~amin & amax & bmax;
\end{lstlisting}
to find the bits not triggered by \eqref{011maxor} where \eqref{010maxor} 
occurs. 
Assuming $xi1 \ne 0$, $i_1 = bsr(xi1)$.
\begin{lstlisting}
    uint32_t xi01 = ((1 << bsr(xi0|1))-1) & amax & bmax;
\end{lstlisting}
to find the bits modified by \eqref{011maxor} which trigger \eqref{010maxor}.

Assuming $xi01 \ne 0$, $i_a = max(i_0, bsr(xi01))$. Without using $max$,
this can be computed as follows:
\begin{lstlisting}
    uint32_t xa = xi1 | xi01;
    uint32_t ia = bsr(xa|1);
    return ia;
}
\end{lstlisting}

Altogether, the algorithm expressed in C is as follows:
\begin{lstlisting}
#include <stdint.h>
uint32_t maxor(uint32_t amin, uint32_t amax, uint32_t bmin, uint32_t bmax){
    uint32_t xa1 = ~amin&amax&bmax;
    uint32_t xa0 = ~amin&amax&~bmax;
    uint32_t xb1 = ~bmin&bmax&amax;
    uint32_t xb0 = ~bmin&bmax&~amax;
    xa1 |= ((1 << bsr(xa0|1))-1)&amax&bmax;
    xb1 |= ((1 << bsr(xb0|1))-1)&amax&bmax;
    if(xa1 > xb1) amax |= (1 << bsr(xa1|1))-1;
    else if(xb1 > 0) bmax |= (1 << bsr(xb1|1))-1;
    return amax|bmax;
}
\end{lstlisting}

Note that 

\begin{lstlisting}
    cmax = (amax|amask)|(bmax|bmask);
\end{lstlisting}

Can be rearranged to

\begin{lstlisting}
    cmax = (amax|bmax)|(amask|bmask);
\end{lstlisting}

Thus we can reduce the code a bit to
\begin{lstlisting}
#include <stdint.h>
uint32_t maxor(uint32_t amin, uint32_t amax, uint32_t bmin, uint32_t bmax){
    uint32_t x1 = (~amin&amax& bmax)|(~bmin&bmax& amax);
    uint32_t x0 = (~amin&amax&~bmax)|(~bmin&bmax&~amax);
    x1 |= ((1 << bsr(x0|1))-1)&amax&bmax;
    return amax|bmax|((1 << bsr(x1|1))-1);
}
\end{lstlisting}


\section{unsigned $min \{a\&b\}$}

The same basic construction as \ref{maxand} may be used. 
The interesting differences are:

It follows from \eqref{unsdef} that for $\alpha \in a^i$, $\beta \in b^i$:
\begin{equation}
(\alpha \; \text{\&} \; \beta)_i = 0
\quad \Rightarrow \quad c_{min,i} = 0
\end{equation}

then from \eqref{dn1}, it follows that 
\begin{equation}\label{minand}
c_{min,i} = a^i_{min,i} \; \text{\&}\; b^i_{min,i}
\end{equation}

then
\begin{equation*}
\eqref{01} \rightarrow c_{min,i} = 0
\end{equation*}

By \eqref{minand},
\begin{equation*}
\alpha_i = 0 \vee \beta_i = 0 \quad \Rightarrow \quad c_{min,i} = 0
\end{equation*}
Note the similarity to max\{a\textbar b\}, which is
\begin{equation*}
\alpha_i = 1 \vee \beta_i = 1 \quad \Rightarrow \quad c_{max,i} = 1
\end{equation*}

Define 
\begin{equation*}
\sim a = [\sim a_{max}, \sim a_{min}]
\end{equation*}
Note that $(\sim a)_{min} \le (\sim a)_{max}$ holds. 
If $a_{min} = a_{max}$, then 
$(\sim a)_{min} = (\sim a)_{max}$.
For $a_{min} < a_{max}$, consider the highest bit $i$ where 
$a_{min,i} \ne a_{max,i}$. By \eqref{01}, it must be that 
$a_{min,i} = 0 \wedge a_{max,i} = 1$. 
Then
$(\sim a)_{min,i} = 0 \wedge (\sim a)_{max,i} = 1$, and 
$(\sim a)_{min} < (\sim a)_{max}$. 

Define
\begin{equation*}
c = max\{\sim a \text{\textbar} \sim b\}
\end{equation*}
Then for $\alpha \in (\sim a)$, $\beta \in (\sim b)$,
\begin{equation*}
\alpha_i = 1 \vee \beta_i = 1 \quad \Leftrightarrow \quad 
(\sim \alpha)_i = 0 \vee (\sim \beta)_i = 0
\end{equation*}
\begin{equation*}
(\sim \alpha)_i = 0 \vee (\sim \beta)_i = 0 \quad \Rightarrow \quad 
(c_i = 1 \quad \Leftrightarrow \quad (\sim c)_i = 0)
\end{equation*}
\begin{equation*}
(\sim \alpha) \in a
\end{equation*}
then
\begin{equation}
min\{a\text{\&}b\} = 
\sim max\{\sim a \text{\textbar} \sim b\}
\end{equation}

\begin{comment}
For $\eqref{01} \wedge b^i_{min,i} = 0$, $\check a_i $ is unconstrained.
Since $c_{min}$ will be minimized when $a^j_{min,j} = 0$,
$\check a_i = 1$ is desireable. Then for

\begin{equation}\label{011minand} 
a^{i-1} \quad = \quad [up(a^i_{min}, i), a^i_{max}]
\end{equation}

$\check a \in a^{i-1}$, and \eqref{le1} holds because 
$a^{i-1}_{min,j} = 0$ for $j<i$. Likewise for $b$.
\vspace{10pt}

For $\eqref{01} \wedge b^i_{min,i} = 1$, it must be that $\check a_i = 0$. Then for
\begin{equation}\label{010minand} 
a^{i-1} \quad = \quad [a^i_{min}, dn(a^i_{max},i)]
\end{equation}

$\check a \in a^{i-1}$, and \eqref{le1} holds because 
$a^{i-1}_{max,j} = 1$ for $j<i$. Likewise for $b$.
\vspace{10pt}
Now it is \eqref{011minand} that affects the result, and we must find
the first bit on which it is applied.
\vspace{10pt}

\begin{tabular}{l|c|c|c|c|l r}
& $a_{min,i}$ & $a_{max,i}$ & $b_{min,i}$ & $b_{max,i}$ & effect &\\
\hline
$\eqref{011minand}_a$ & $0$ & $1$ & $0$ & & $a_{min,j} := 0$, \;& $j < i$ \\
\hline
$\eqref{011minand}_b$ & $0$ & & $0$ & $1$ & $b_{min,j} := 0$, \;& $j < i$ \\
\hline
$\eqref{010minand}_a$ & $0$ & $1$ & $1$ & & $a_{max,j} := 1$, \;& $j < i$ \\
\hline
$\eqref{010minand}_b$ & $1$ & & $0$ & $1$ & $b_{max,j} := 1$, \;& $j < i$ \\
\end{tabular}
\vspace{10pt}

Then\\
$\eqref{011minand}_a$ \quad can trigger \quad $\eqref{011minand}_b$,
$\eqref{010minand}_a$\\
$\eqref{011minand}_b$ \quad can trigger \quad $\eqref{011minand}_a$,
$\eqref{010minand}_b$\\
$\eqref{010minand}_a$ \quad can trigger \quad $\eqref{011minand}_a$ \\
$\eqref{010minand}_b$ \quad can trigger \quad $\eqref{011minand}_b$ \\

Because $\eqref{011minand}_b$ sets $b_{min} = 0$ for following bits, then by
\eqref{minand}, $c_{min,j} = 0$ regardless of $a_{min}$.
This makes any
application of $\eqref{011minand}_a$ occurring at bit $j<i$ superfluous. 
Then we need only find the first of $\eqref{011minand}_a$ and 
$\eqref{011minand}_b$.
For $\eqref{011minand}_a$, we need consider
\begin{equation*} 
\eqref{010minand}_a \rightarrow \eqref{011minand}_a
\end{equation*}

Therefore, 
the algorithm for min\{a\&b\} is:

Given ranges $a$ and $b$, define $i_a$ as the index of the first application 
of $\eqref{011minand}_a$.

\begin{equation*}
    i_a = 
    max \, ( \, \{i_0\} \cup 
    \{ i \in I_{i_1} \; : \; aabb_i = 0*0*\} 
    )
\end{equation*}

    where
\begin{equation*}
i_0 = max \, \{i \in I_N\; : \; 
aabb_i = 010*\},\,
\text{or $-1$ if the set is empty}
\end{equation*}
\begin{equation*}
i_1 = max \, \{i\in I_N\; : \; 
aabb_i = 011*\},\,
\text{or $-1$ if the set is empty}
\end{equation*}
\vspace{0pt}

if $i_a > i_b$, then

\quad $c_{min} = up\,(a_{min},i_a) \; \text{\&} \; b_{min} $

otherwise if $i_b > -1$,

\quad $c_{min} = a_{min} \; \text{\&} \; up\,(b_{min},i_b) $

otherwise,

\quad $c_{min} = a_{min} \; \text{\&} \; b_{min} $

\vspace{20pt}
Keeping in mind the optimizations explained circa \ref{oho}, 
given ranges $a$ and $b$:
we can compute $i_a$ as follows:
\begin{lstlisting}
#include<stdint.h>
uint32_t i_a(uint32_t amin, uint32_t amax, uint32_t bmin){
    uint32_t xi0 = ~amin & amax & ~bmin;
\end{lstlisting}
to find the bits unmodified by \eqref{010minand} in which \eqref{011minand} 
occurs.  Then when $i_0 \ne -1$, $i_0 = bsr(xi0)$.
\begin{lstlisting}
    uint32_t xi1 = ~amin & amax & bmin;
\end{lstlisting}
to find the bits in which \eqref{010minand} 
occurs. 
Then when $i_1 \ne -1$, $i_1 = bsr(xi1)$.
\begin{lstlisting}
    uint32_t xi01 = ((1 << bsr(xi1|1))-1) & ~amin & ~bmin;
\end{lstlisting}
to find the bits modified by \eqref{010minand} in which \eqref{011minand} occur.

When $i_a \ne -1$, $i_a = max(i_0, bsr(xi01))$. Without using $max$,
this can be computed as follows:
\begin{lstlisting}
    uint32_t xa = xi0 | xi01; 
    uint32_t ia = bsr(xa|1);
    return ia;
}
\end{lstlisting}

Then, to compute $up(a_{min}, i_a)$:
\begin{lstlisting}
uint32_t up(uint32_t amin, uint32_t amax, uint32_t bmin){
    uint32_t mask = (1 << i_a(amin,amax,bmin))-1;
    return amin&~mask;
}
\end{lstlisting}

Again, setting $a_{min,i} = 1$ has no effect on $c_{min}$, so it is not
performed.

Altogether, the algorithm expressed in C is as follows:
\begin{lstlisting}
#include <stdint.h>
uint32_t minand(uint32_t amin, uint32_t amax, uint32_t bmin, uint32_t bmax){
    uint32_t xa1 = ~amin& amax& bmin;
    uint32_t xa0 = ~amin& amax&~bmin;
    uint32_t xb1 =  amin&     &~bmin& bmax;
    uint32_t xb0 = ~amin&     &~bmin& bmax;
    xa0 |= ((1 << bsr(xa1|1))-1)&~amin&~bmin;
    xb0 |= ((1 << bsr(xb1|1))-1)&~amin&~bmin;
    if(xa0 > xb0) amin &= ~((1 << bsr(xa0))-1);
    else bmin &= ~((1 << bsr(xb0))-1);
    return amin&bmin;
}
\end{lstlisting}

Note that 

\begin{lstlisting}
cmin = (amin&~amask)&(bmin&~bmask);
\end{lstlisting}

can be rearranged to

\begin{lstlisting}
cmin = (amin&bmin)&~(amask|bmask);
\end{lstlisting}

Thus, we can reduce the algorithm a bit to
\begin{lstlisting}
#include <stdint.h>
uint32_t minand(uint32_t amin, uint32_t amax, uint32_t bmin, uint32_t bmax){
    uint32_t x1 = (~amin&amax& bmin)|(~bmin&bmax& amin);
    uint32_t x0 = (~amin&amax&~bmin)|(~bmin&bmax&~amin);
    uint32_t imask = x0 | (((1 << bsr(x1|1))-1)&~amin&~bmin);
    return amin&bmin & ~((1 << bsr(imask|1))-1);
}
\end{lstlisting}
\end{comment}

\section{unsigned $min\{a\text{\textbar} b\}$}

Again, the same basic construction as \ref{maxand} may be used. 
The interesting differences are:

It follows from \eqref{unsdef} that for $\alpha \in a^i$, $\beta \in b^i$:
\begin{equation}
(\alpha \; \text{\textbar} \; \beta)_i = 0
\quad \Rightarrow \quad c_{min,i} = 0
\end{equation}

then

\begin{equation*}
(\alpha_i = 0  \wedge \beta_i = 0) \quad \Rightarrow \quad 
c_{min,i} = 0
\end{equation*}

Note the similarity to $max\{a\&b\}$, where

\begin{equation*}
(\alpha_i = 1  \wedge \beta_i = 1) \quad \Rightarrow \quad 
c_{max,i} = 1
\end{equation*}

Define
\begin{equation*}
c = max\{\sim a \, \& \sim b\}
\end{equation*}
Then for $\alpha \in (\sim a)$, $\beta \in (\sim b)$,
\begin{equation*}
\alpha_i = 1 \wedge \beta_i = 1 \quad \Leftrightarrow \quad 
(\sim \alpha)_i = 0 \wedge (\sim \beta)_i = 0
\end{equation*}
\begin{equation*}
(\sim \alpha)_i = 0 \wedge (\sim \beta)_i = 0 \quad \Rightarrow \quad 
(c_i = 1 \quad \Leftrightarrow \quad (\sim c)_i = 0)
\end{equation*}
\begin{equation*}
(\sim \alpha) \in a, (\sim \beta) \in b
\end{equation*}
then
\begin{equation}
min\{a\text{\textbar}b\} = \,
\sim max\{\sim a \, \& \sim b\}
\end{equation}

\begin{comment}
then from \eqref{up1}, it follows that 
\begin{equation}\label{minor}
c_{min,i} = a^i_{min,i} \; \text{\textbar}\; b^i_{min,i}
\end{equation}

Then
\begin{equation*}
\eqref{01} \quad \Rightarrow \quad 
(c_{min,i} = 1 \quad \Leftrightarrow \quad b^i_{min,i} = 1)
\end{equation*}
For $\eqref{01} \wedge b^i_{min,i} = 1$, then $\check a_i$ is unconstrained.
Since $c_{min}$ is minimized when $a^j_{min,j} = 0$,
$\check a_i = 1$ is desirable.
Then 
\begin{equation}\label{011minor} 
a^{i-1} \quad = \quad [up(a^i_{min}, i), a^i_{max}]
\end{equation}

Likewise for $b$.
\vspace{10pt}

For $\eqref{01} \wedge b^i_{min,i} = 0$, $\check a_i = 0$ must be true.
Then
\begin{equation}\label{010minor} 
a^{i+1} \quad = \quad [a^i_{min}, dn(a^i_{max},i)]
\end{equation}

Likewise for $b$.
\vspace{10pt}

Now it is \eqref{011minor} that affects the result, so
the first occurrence of it must be found.
\vspace{10pt}

\begin{tabular}{l|c|c|c|c|l r}
& $a_{min,i}$ & $a_{max,i}$ & $b_{min,i}$ & $b_{max,i}$ & effect &\\
\hline
$\eqref{011minor}_a$ & $0$ & $1$ & $1$ & & $a_{min,j} := 0$, \;& $j < i$ \\
\hline
$\eqref{011minor}_b$ & $1$ & & $0$ & $1$ & $b_{min,j} := 0$, \;& $j < i$ \\
\hline
$\eqref{010minor}_a$ & $0$ & $1$ & $0$ & & $a_{max,j} := 1$, \;& $j < i$ \\
\hline
$\eqref{010minor}_b$ & $0$ & & $0$ & $1$ & $b_{max,j} := 1$, \;& $j < i$ \\
\end{tabular}
\vspace{10pt}

Then\\
$\eqref{011minor}_a$ \quad can trigger \quad $\eqref{010minor}_a$,
$\eqref{010minor}_b$\\
$\eqref{011minor}_b$ \quad can trigger \quad $\eqref{010minor}_a$,
$\eqref{010minor}_b$\\
$\eqref{010minor}_a$ \quad can trigger \quad $\eqref{011minor}_a$ \\
$\eqref{010minor}_b$ \quad can trigger \quad $\eqref{011minor}_b$ \\

It can be seen that $\eqref{011minor}_a$ cannot succeed $\eqref{011minor}_b$, 
and vice versa.
Then the possible dependency chains for $\eqref{011minor}_a$ are
\begin{equation*} 
\eqref{010minor}_a \rightarrow \eqref{011minor}_a
\end{equation*}

Therefore, the algorithm may be rewritten as follows:

Given ranges $a$ and $b$, define $i_a$ as the index of the first occurrence
of $\eqref{011minor}_a$.
\begin{equation*}
i_a = max \, ( \, \{i_1\} \cup \{i \in I_{i_0}\; : \; 
        aabb_i = 0*1*\} \, ) 
\end{equation*}
    where
\begin{equation*}
i_0 = max \, \{i \in I_N\; : \; 
aabb_i = 010*\},\,
\text{or $-1$ if the set is empty}
\end{equation*}
\begin{equation*}
i_1 = max \, \{i\in I_N\; : \; 
aabb_i = 011*\},\,
\text{or $-1$ if the set is empty}
\end{equation*}
\vspace{0pt}

if $i_a > i_b$, then

\quad $c_{min} = up\,(a_{min},i_a) \; \text{\&} \; b_{min} $

otherwise if $i_b > -1$,

\quad $c_{min} = a_{min} \; \text{\&} \; up\,(b_{min},i_b) $

otherwise,

\quad $c_{min} = a_{min} \; \text{\&} \; b_{min} $

Keeping in mind the optimizations explained circa \ref{oho}, 
we can compute $i_a$ as follows:
\begin{lstlisting}
#include<stdint.h>
uint32_t i_a(uint32_t amin, uint32_t amax, uint32_t bmin){
    uint32_t xi0 = ~amin & amax & ~bmin;
\end{lstlisting}
to find the bits at which \eqref{010minor} occurs.
When $i_0 \ne -1$, $i_0 = bsr(xi0)$.
\begin{lstlisting}
    uint32_t xi1 = ~amin & amax & bmin;
\end{lstlisting}
to find the bits unmodified by \eqref{010minor} at which \eqref{011minor} 
occurs. 
When $i_1 \ne -1$, $i_1 = bsr(xi1)$.
\begin{lstlisting}
    uint32_t xi01 = ((1 << bsr(xi1|1))-1) & ~amin & bmin;
\end{lstlisting}
to find the bits modified by \eqref{010minor} at which \eqref{011minor} occurs.

When $i_a \ne -1$, $i_a = max(i_0, bsr(xi01))$. Without using $max$,
this can be computed as follows:
\begin{lstlisting}
    uint32_t xa = xi0 | xi01;
    uint32_t ia = bsr(xa|1);
    return ia;
}
\end{lstlisting}

Altogether, the algorithm expressed in C is as follows:
\begin{lstlisting}
#include <stdint.h>
uint32_t maxand(uint32_t amin, uint32_t amax, uint32_t bmin, uint32_t bmax){
    uint32_t xa1 = ~amin&amax& bmin;
    uint32_t xa0 = ~amin&amax&~bmin;
    uint32_t xb1 =  amin&    &~bmin& bmax;
    uint32_t xb0 = ~amin&    &~bmin& bmax;
    xa1 |= ((1 << bsr(xa0|1))-1)&~amin&bmin;
    xb1 |= ((1 << bsr(xb0|1))-1)&~bmin&amin;
    if(xa1 > xb1) amin &= ~((1 << bsr(xa1|1))-1);
    else  bmin &= ~((1 << bsr(xb1|1))-1);
    return amin|bmin;
}
\end{lstlisting}
\end{comment}

\section{unsigned $max\{a$\^{}$b\}$}

Again, the same basic construction as \ref{maxand} may be used. 
The interesting differences are:

It follows from \eqref{unsdef} that for $\alpha \in a^i$, $\beta \in b^i$:
\begin{equation}
(\alpha \; \text{\^{}} \; \beta)_i = 1
\quad \Rightarrow \quad c_{max,i} = 1
\end{equation}

then this one is tricky. Decomposed:
\begin{equation*}
(\alpha_i = 1 \wedge \beta_i = 0) \vee (\alpha_i = 0 \wedge \beta_i = 1) 
\quad \Rightarrow \quad c_{max,i} = 1
\end{equation*}

note that
\begin{equation}\label{dn1}
a^i_{min, i} = 0 \quad \Rightarrow \quad \alpha_i = 0 \quad 
\text{for some} \; \alpha \in a^i
\end{equation}
Then by \eqref{up1} and \eqref{dn1},
\begin{equation}\label{maxxor}
c_{max,i} = (a^i_{max,i} \; \text{\^{}}\; b^i_{min,i}) 
\text{\textbar} (a^i_{min,i} \; \text{\^{}}\; b^i_{max,i}) 
\end{equation}

Then by \eqref{le1} through \eqref{11}, it can be shown that
\begin{equation*}
\eqref{01} \quad \Rightarrow \quad 
c_{max,i} = 1 
\end{equation*}

For $\eqref{01} \wedge b^i_{max,i} = 0$, 
then $ \hat a_i = 1$ must be true.
Then
\begin{equation}\label{0100maxxor} 
a^{i-1} \quad = \quad [up(a^i_{min}, i), a^i_{max}]
\end{equation}

Likewise for $b$.
\vspace{10pt}

For $\eqref{01}  \wedge  b^i_{min,i} = 1$, 
then $\hat a_i = 0$ must be true.
Then
\begin{equation}\label{0111maxxor} 
a^{i-1} \quad = \quad [a^i_{min}, dn(a^i_{max},i)]
\end{equation}

Likewise for $b$.
\vspace{10pt}

For $\eqref{01}_a  \wedge  \eqref{01}_b$, there is no imposed constraint.
In order to maximize $c_{max,j}$ for $j<i$, 
we choose $\hat a_i = 0$, and $\hat b_i = 1$.
Then
\begin{equation}\label{0101maxxor}
\begin{tabular}{c}
$a^{i-1} \quad = \quad [a^i_{min}, dn(a^i_{max},i)]$\\
$b^{i-1} \quad = \quad [up(b^i_{min},i), b^i_{max}]$
\end{tabular}
\end{equation}

Now any of \eqref{0100maxxor}, \eqref{0111maxxor}, or \eqref{0101maxxor} 
may affect the result, though \eqref{0101maxxor}
is a special case.
So we must find the first occurrences of 
\eqref{0100maxxor} and \eqref{0111maxxor}.

\begin{tabular}{l|c|c|c|c|l r}
& $a_{min,i}$ & $a_{max,i}$ & $b_{min,i}$ & $b_{max,i}$ & effect &\\
\hline
$\eqref{0100maxxor}_a$ & $0$ & $1$ & $0$ & $0$ & 
$a_{min,j} := 0$, \;& $j < i$ \\
\hline
$\eqref{0100maxxor}_b$ & $0$ & $0$ & $0$ & $1$ & 
$b_{min,j} := 0$, \;& $j < i$ \\
\hline
$\eqref{0111maxxor}_a$ & $0$ & $1$ & $1$ & $1$ & 
$a_{max,j} := 1$, \;& $j < i$ \\
\hline
$\eqref{0111maxxor}_b$ & $1$ & $1$ & $0$ & $1$ & 
$b_{max,j} := 1$, \;& $j < i$ \\
\hline
$\eqref{0101maxxor}$ & $0$ & $1$ & $0$ & $1$ & 
$c_{max,j} := 1$, \;& $j < i$ \\
\end{tabular}
\vspace{10pt}

Then\\
$\eqref{0100maxxor}_a$ \quad can trigger \quad $\eqref{0100maxxor}_b$,
$\eqref{0111maxxor}_a$, $\eqref{0101maxxor}$\\
$\eqref{0100maxxor}_b$ \quad can trigger \quad $\eqref{0100maxxor}_a$,
$\eqref{0111maxxor}_b$, $\eqref{0101maxxor}$\\
$\eqref{0111maxxor}_a$ \quad can trigger \quad $\eqref{0100maxxor}_a$,
$\eqref{0111maxxor}_b$, $\eqref{0101maxxor}$\\
$\eqref{0111maxxor}_b$ \quad can trigger \quad $\eqref{0100maxxor}_b$,
$\eqref{0111maxxor}_a$, $\eqref{0101maxxor}$\\

It can be seen that $\eqref{0111maxxor}_b$ cannot succeed or precede
$\eqref{0100maxxor}_a$. 
Then the possible dependency chains for the first application of 
$\eqref{0100maxxor}_a$ and $\eqref{0111maxxor}_a$ are:
\begin{equation*} 
\eqref{0100maxxor}_b \rightarrow \eqref{0100maxxor}_a
\end{equation*}
\begin{equation*} 
\eqref{0111maxxor}_a \rightarrow \eqref{0100maxxor}_a
\end{equation*}
\begin{equation*} 
\eqref{0100maxxor}_a \rightarrow 
\eqref{0111maxxor}_a
\end{equation*}
\begin{equation*} 
\eqref{0111maxxor}_b \rightarrow 
\eqref{0111maxxor}_a
\end{equation*}

And likewise for $b$.

Any single or pair of applications above can trigger
\eqref{0101maxxor}, and none can invalidate 
\eqref{0101maxxor}.

\vspace{10pt}

Then given ranges $a$ and $b$,
\vspace{10pt}

Define 
\begin{tabular}{l}
$i_a$ as the first bit where $\eqref{0100maxxor}_a$ occurs\\
$j_a$ as the first bit where $\eqref{0111maxxor}_a$ occurs\\
$k$ as the first bit where $\eqref{0101maxxor}$ occurs
\end{tabular}
\begin{equation*}
i_a = 
max \, ( \, \{i_{a00}\} \cup 
\{i \in I_{i_{b00}}\; : \; aabb_i = 01*0\} \, \cup
\{i \in I_{i_{a11}}\; : \; aabb_i = 0*00\} 
) 
\end{equation*}
\begin{equation*}
j_a = 
max \, ( \, \{i_{a11}\} \cup 
\{i \in I_{i_{a00}}\; : \; aabb_i = *111\} \, \cup
\{i \in I_{i_{b11}}\; : \; aabb_i = 011*\} 
) 
\end{equation*}
\begin{equation*}
k = max \, 
\left(
\begin{tabular}{l c}
\huge $\cup$ \normalsize
\begin{tabular}{c}
$\{i_{ab}\}$\\
$\{i \in I_{i_{a00}}\; : \; aabb_i = *101\} $\\
$\{i \in I_{i_{b00}}\; : \; aabb_i = 01*1\} $\\
$\{i \in I_{i_{a11}}\; : \; aabb_i = 0*01\} $\\
$\{i \in I_{i_{b11}}\; : \; aabb_i = 010*\} $\\
\end{tabular}
\end{tabular}
\right)
\end{equation*}

    where
\begin{equation*}
i_{a00} = max \, \{i \in I_N\; : \; 
a_{min,i}, a_{max,i},b_{min,i},b_{max,i} = 0,1,0,0\},\,
\text{or $-1$ if the set is empty}
\end{equation*}
\begin{equation*}
i_{b00} = max \, \{i \in I_N\; : \; 
a_{min,i}, a_{max,i},b_{min,i},b_{max,i} = 0,0,0,1\},\,
\text{or $-1$ if the set is empty}
\end{equation*}
\begin{equation*}
i_{a11} = max \, \{i \in I_N\; : \; 
a_{min,i}, a_{max,i},b_{min,i},b_{max,i} = 0,1,1,1\},\,
\text{or $-1$ if the set is empty}
\end{equation*}
\begin{equation*}
i_{b11} = max \, \{i \in I_N\; : \; 
a_{min,i}, a_{max,i},b_{min,i},b_{max,i} = 1,1,0,1\},\,
\text{or $-1$ if the set is empty}
\end{equation*}
\begin{equation*}
i_{ab} = max \, \{i \in I_N\; : \; 
a_{min,i}, a_{max,i},b_{min,i},b_{max,i} = 0,1,0,1\},\,
\text{or $-1$ if the set is empty}
\end{equation*}


given ranges $a,b$

if $i_a > j_b$, then

\quad $\bar a_{min} = up\,(a_{min},i_a) $

\quad $\bar b_{max} = b_{max}$

otherwise,

\quad $\bar a_{min} = a_{min} $

\quad if $j_b > -1$, then

\quad\quad $\bar b_{max} = dn\,(b_{max},j_b) $

likewise for b

if $k > -1$, then

\quad $m = dn(0,k)$

otherwise,

\quad $m = 0$

$c_{max} = (\bar a_{min} \; \text{\^{}} \; \bar b_{max})\;|\;(\bar a_{max} \; 
\text{\^{}} \; \bar b_{min})\;|\; m $

\vspace{10pt}

Then in C, given ranges $a$ and $b$,
keeping in mind the optimizations explained circa \ref{oho}:

\begin{lstlisting}
uint32_t maxxor(uint32_t amin, uint32_t amax, uint32_t bmin, uint32_t bmax){
    uint32_t xa00 = (~amin& amax&~bmin&~bmax);
    uint32_t xb00 = (~amin&~amax&~bmin& bmax);
    uint32_t xa11 = (~amin& amax& bmin& bmax);
    uint32_t xb11 = ( amin& amax&~bmin& bmax);
    uint32_t xab  = (~amin& amax&~bmin& bmax);
    uint32_t xia  = xa00 | (((1 << bsr(xb00|1))-1)&~amin& amax&~bmax) 
                         | (((1 << bsr(xa11|1))-1)&~amin&~bmin&~bmax);
    uint32_t xib  = xb00 | (((1 << bsr(xa00|1))-1)&~bmin& bmax&~amax) 
                         | (((1 << bsr(xb11|1))-1)&~bmin&~amin&~amax);
    uint32_t xja  = xa11 | (((1 << bsr(xa00|1))-1)& amax& bmin& bmax) 
                         | (((1 << bsr(xb11|1))-1)&~amin& amax& bmin);
    uint32_t xjb  = xb11 | (((1 << bsr(xb00|1))-1)& bmax& amin& amax) 
                         | (((1 << bsr(xa11|1))-1)&~bmin& bmax& amin);
    uint32_t xk   = xab  | (((1 << bsr(xia |1))-1)& amax&~bmin& bmax)
                         | (((1 << bsr(xib |1))-1)& bmax&~amin& amax)
                         | (((1 << bsr(xja |1))-1)&~amin& amax&~bmin)
                         | (((1 << bsr(xjb |1))-1)&~bmin& bmax&~amin);
    // todo: why do these all have to run always?
    amin &= ~((1 << bsr(xia|1))-1);
    bmax |=  ((1 << bsr(xjb|1))-1);
    bmin &= ~((1 << bsr(xib|1))-1);
    amax |=  ((1 << bsr(xja|1))-1);

    return (amin ^ bmax) | (amax ^ bmin) | ((1 << bsr(xk|1))-1);
}
\end{lstlisting}

\section{unsigned $min\{a$\^{}$b\}$}

Again, the same basic construction as \ref{maxand} may be used for this one. 
The interesting differences are:

It follows from \eqref{unsdef} that for $\alpha \in a$, $\beta \in b$:
\begin{equation}
(\alpha \; \text{\^{}} \; \beta)_i = 0
\quad \Rightarrow \quad c_{min,i} = 0
\end{equation}

then:
\begin{equation*}
(\alpha_i = 0 \wedge \beta_i = 0) \vee (\alpha_i = 1 \wedge \beta_i = 1) 
\quad \Rightarrow \quad c_{min,i} = 0
\end{equation*}

By \eqref{up1} and \eqref{dn1},
\begin{equation*}
(a_{min,i} = 0 \wedge b_{min,i} = 0) \vee (a_{max,i} = 1 \wedge b_{max,i} = 1) 
\quad \Rightarrow \quad c_{min,i} = 0
\end{equation*}
Then:
\begin{equation}\label{minxor}
c_{min,i} = (a_{min,i} \; \text{\^{}}\; b_{min,i}) 
\text{\&} (a_{max,i} \; \text{\^{}}\; b_{max,i}) 
\end{equation}

By \eqref{le1}, it can be shown that
\begin{equation*}
\eqref{01} \quad \Rightarrow \quad 
c_{min,i} = 0 
\end{equation*}
For $\eqref{01} \wedge b_{min,i} = 1 \wedge b_{max,i} = 1$, 
then $ \check a_i = 1$ must be true.
Therefore the range $[up(a_{min}, i), a_{max}]$ certainly contains $\check a$.

Then, for this case, 
\begin{equation}\label{0111minxor} 
a \quad \text{becomes} \quad [up(a_{min}, i), a_{max}]
\end{equation}

Note that now \eqref{le1} holds for any $j < i$ because $a_{min,j} = 0$. 

Likewise for $b$.\\
For $\eqref{01}  \wedge  b_{min,i} = 0 \wedge b_{max,i} = 0$, 
then $\check a_i = 0$ must be true.
Therefore the range $[a_{min}, dn(a_{max},i)]$ certainly contains $\check a$.

Then, for this case, 
\begin{equation}\label{0100minxor} 
a \quad \text{becomes} \quad [a_{min}, dn(a_{max},i)]
\end{equation}

Note that now \eqref{le1} holds for any $j < i$, because $a_{max,j} = 1$.

Likewise for $b$.\\
For $\eqref{01}_a  \wedge  \eqref{01}_b$, there is no imposed constraint.
In order to minimize $c_{max,j}$ for $j<i$, 
we choose to constrain $\hat a_i = 0$, and $\hat b_i = 0$.
Then, for this case, 
\begin{equation}\label{0101minxor}
\begin{tabular}{c}
$a \quad \text{becomes} \quad [a_{min}, dn(a_{max},i)]$\\
$b \quad \text{becomes} \quad [b_{min}, dn(b_{max},i)]$
\end{tabular}
\end{equation}

Note that now \eqref{le1} holds for any $j < i$, because $a_{max,j} = 1$ and likewise for $b$.

Now any of \eqref{0111minxor}, \eqref{0100minxor}, or \eqref{0101minxor} 
may affect the result, though \eqref{0101minxor}
is a special case.
So the ways of triggering both
\eqref{0111minxor} and \eqref{0100minxor}
must be considered. 
\vspace{10pt}

\begin{tabular}{l|c|c|c|c|l r}
& $a_{min,i}$ & $a_{max,i}$ & $b_{min,i}$ & $b_{max,i}$ & effect &\\
\hline
$\eqref{0111minxor}_a$ & $0$ & $1$ & $1$ & $1$ & 
$a_{min,j} := 0$, \;& $j < i$ \\
\hline
$\eqref{0111minxor}_b$ & $1$ & $1$ & $0$ & $1$ & 
$b_{min,j} := 0$, \;& $j < i$ \\
\hline
$\eqref{0100minxor}_a$ & $0$ & $1$ & $0$ & $0$ & 
$a_{max,j} := 1$, \;& $j < i$ \\
\hline
$\eqref{0100minxor}_b$ & $0$ & $0$ & $0$ & $1$ & 
$b_{max,j} := 1$, \;& $j < i$ \\
\hline
$\eqref{0101minxor}$ & $0$ & $1$ & $0$ & $1$ & 
$c_{min,j} := 0$, \;& $j < i$ \\
\end{tabular}
\vspace{10pt}

Then\\
$\eqref{0111minxor}_a$ \quad can trigger \quad $\eqref{0100minxor}_a$,
$\eqref{0100minxor}_b$, $\eqref{0101minxor}$\\
$\eqref{0111minxor}_b$ \quad can trigger \quad $\eqref{0100minxor}_a$,
$\eqref{0100minxor}_b$, $\eqref{0101minxor}$\\
$\eqref{0100minxor}_a$ \quad can trigger \quad $\eqref{0111minxor}_a$,
$\eqref{0111minxor}_b$, $\eqref{0101minxor}$\\
$\eqref{0100minxor}_b$ \quad can trigger \quad $\eqref{0111minxor}_a$,
$\eqref{0111minxor}_b$, $\eqref{0101minxor}$\\

Note that $\eqref{0100minxor}_b$ will cause $b_{max} = 1$ for 
following bits, and
$\eqref{0100minxor}_a$ at $j$ requires $b_{max,j} = 0$, and
$\eqref{0111minxor}_b$ will cause $b_{min} = 0$ for following bits
and $\eqref{0111minxor}_a$ requires $b_{min} = 1$, and vice versa,
therefore 
$\eqref{0111minxor}_a$ and
$\eqref{0111minxor}_b$ cannot coexist in the same dependency chain, and
neither can $\eqref{0100minxor}_a$ and $\eqref{0100minxor}_b$.
Then the possible dependency chains for the first application of 
$\eqref{0111minxor}_a$ and $\eqref{0100minxor}_a$ are:
\begin{equation*} 
\eqref{0100minxor}_a \rightarrow 
\eqref{0111minxor}_a
\end{equation*}
\begin{equation*} 
\eqref{0100minxor}_b \rightarrow 
\eqref{0111minxor}_a
\end{equation*}
\begin{equation*} 
\eqref{0111minxor}_a \rightarrow 
\eqref{0100minxor}_a
\end{equation*}
\begin{equation*} 
\eqref{0111minxor}_b \rightarrow 
\eqref{0100minxor}_a
\end{equation*}

And likewise for $b$.

Any single or pair of applications above can trigger
\eqref{0101minxor}, and none can invalidate 
\eqref{0101minxor}.

\vspace{10pt}

Then: for range $a$,
\vspace{10pt}

Define 
\begin{tabular}{l}
$i_a$ as the first bit where $\eqref{0111minxor}_a$ occurs\\
$j_a$ as the first bit where $\eqref{0100minxor}_a$ occurs\\
$k$ as the first bit where $\eqref{0101minxor}$ occurs
\end{tabular}
\begin{equation*}
i_a = 
max \, ( \, \{i_{a11}\} \cup 
\{i \in I_{i_{a00}}\; : \; a_{min,i},b_{min,i},b_{max,i} = 0,1,1\} \, \cup
\{i \in I_{i_{b00}}\; : \; a_{min,i},a_{max,i},b_{min,i} = 0,1,1\} 
) 
\end{equation*}
\begin{equation*}
j_a = 
max \, ( \, \{i_{a00}\} \cup 
\{i \in I_{i_{a11}}\; : \; a_{max,i},b_{min,i},b_{max,i} = 1,0,0\} \, \cup
\{i \in I_{i_{b11}}\; : \; a_{min,i},a_{max,i},b_{max,i} = 0,1,0\} 
) 
\end{equation*}
\begin{equation*}
k = max \, 
\left(
\begin{tabular}{l c}
\huge $\cup$ \normalsize
\begin{tabular}{c}
$\{i_{ab}\}$\\
$\{i \in I_{i_{a11}}\; : \; a_{max,i},b_{min,i},b_{max,i} = 1,0,1\} $\\
$\{i \in I_{i_{b}}\; : \; a_{min,i},a_{max,i},b_{max,i} = 0,1,1\} $\\
$\{i \in I_{j_{a}}\; : \; a_{min,i},b_{min,i},b_{max,i} = 0,0,1\} $\\
$\{i \in I_{j_{b}}\; : \; a_{min,i},a_{max,i},b_{min,i} = 0,1,0\} $\\
\end{tabular}
\end{tabular}
\right)
\end{equation*}

    where
\begin{equation*}
i_{a00} = max \, \{i \in I_N\; : \; 
a_{min,i}, a_{max,i},b_{min,i},b_{max,i} = 0,1,0,0\},\,
\text{or $-1$ if the set is empty}
\end{equation*}
\begin{equation*}
i_{b00} = max \, \{i \in I_N\; : \; 
a_{min,i}, a_{max,i},b_{min,i},b_{max,i} = 0,0,0,1\},\,
\text{or $-1$ if the set is empty}
\end{equation*}
\begin{equation*}
i_{a11} = max \, \{i \in I_N\; : \; 
a_{min,i}, a_{max,i},b_{min,i},b_{max,i} = 0,1,1,1\},\,
\text{or $-1$ if the set is empty}
\end{equation*}
\begin{equation*}
i_{b11} = max \, \{i \in I_N\; : \; 
a_{min,i}, a_{max,i},b_{min,i},b_{max,i} = 1,1,0,1\},\,
\text{or $-1$ if the set is empty}
\end{equation*}
\begin{equation*}
i_{ab} = max \, \{i \in I_N\; : \; 
a_{min,i}, a_{max,i},b_{min,i},b_{max,i} = 0,1,0,1\},\,
\text{or $-1$ if the set is empty}
\end{equation*}


given ranges $a,b$

if $i_a > i_b$, then

\quad $\bar a_{min} = up\,(a_{min},i_a) $

\quad $\bar b_{min} = b_{min}$

otherwise,

\quad $\bar a_{min} = a_{min} $

\quad if $i_b > -1$, then

\quad\quad $\bar b_{min} = up\,(b_{min},i_b) $

if $j_a > j_b$, then

\quad $\bar a_{max} = dn\,(a_{max},j_a) $

\quad $\bar b_{max} = b_{max}$

otherwise,

\quad $\bar a_{max} = a_{max} $

\quad if $j_b > -1$, then

\quad\quad $\bar b_{max} = dn\,(b_{max},j_b) $

if $k > -1$, then

\quad $m = dn(0,k)$

otherwise,

\quad $m = 0$

$c_{min} = (\bar a_{min} \; \text{\^{}} \; \bar b_{min})\;\&\;(\bar a_{max} \; 
\text{\^{}} \; \bar b_{max})\;\&\; \text{\~{}} m $

\vspace{10pt}

Then in C, given ranges $a$ and $b$,
keeping in mind the optimizations explained circa \ref{oho}:
\begin{lstlisting}
uint32_t minxor(uint32_t amin, uint32_t amax, uint32_t bmin, uint32_t bmax){
    uint32_t xa00 = (~amin& amax&~bmin&~bmax);
    uint32_t xb00 = (~amin&~amax&~bmin& bmax);
    uint32_t xa11 = (~amin& amax& bmin& bmax);
    uint32_t xb11 = ( amin& amax&~bmin& bmax);
    uint32_t xab  = (~amin& amax&~bmin& bmax);
    uint32_t xa00a11 = (((1 << bsr(xa00|1))-1)&~amin&       bmin& bmax);
    uint32_t xa00b11 = (((1 << bsr(xa00|1))-1)& amin&      ~bmin& bmax);
    uint32_t xb00a11 = (((1 << bsr(xb00|1))-1)&~amin& amax& bmin      );
    uint32_t xb00b11 = (((1 << bsr(xb00|1))-1)& amin& amax&~bmin      );
    uint32_t xa11a00 = (((1 << bsr(xa11|1))-1)      & amax&~bmin&~bmax);
    uint32_t xb11a00 = (((1 << bsr(xb11|1))-1)&~amin& amax      &~bmax);
    uint32_t xa11b00 = (((1 << bsr(xa11|1))-1)      &~amax&~bmin& bmax);
    uint32_t xb11b00 = (((1 << bsr(xb11|1))-1)&~amin&~amax      & bmax);
    uint32_t xia  = xa11 | xa00a11 | xb00a11;
    uint32_t xib  = xb11 | xa00b11 | xb00b11;
    uint32_t xja  = xa00 | xa11a00 | xb11a00; 
    uint32_t xjb  = xb00 | xa11b00 | xb11b00;

    uint32_t xk   = xab  | (((1 << bsr(xa00|1))-1)   &~amin&      ~bmin& bmax)
                         | (((1 << bsr(xb00|1))-1)   &~amin& amax&~bmin)
                         | (((1 << bsr(xa11|1))-1)         & amax&~bmin& bmax)
                         | (((1 << bsr(xb11|1))-1)   &~amin& amax      & bmax)

                         | (((1 << bsr(xa00a11|xa11a00|1))-1)            &~bmin& bmax)
                         | (((1 << bsr(xa00b11|xb11a00|1))-1)&~amin            & bmax)
                         | (((1 << bsr(xb00a11|xa11b00|1))-1)      & amax&~bmin      )
                         | (((1 << bsr(xb00b11|xb11b00|1))-1)&~amin& amax            );
                     
    if (xia > xib){
        amin &= ~((1 << bsr(xia|1))-1);
    }else{
        bmin &= ~((1 << bsr(xib|1))-1);
    }
    if (xja > xjb){
        amax |=  ((1 << bsr(xja|1))-1);
    }else {
        bmax |=  ((1 << bsr(xjb|1))-1);
    }

    return (amin ^ bmin) & (amax ^ bmax) & ~((1 << bsr(xk|1))-1);
}
\end{lstlisting}

\section{signed max \{a\&b\}}
To maximize a signed range is somewhat more 
complicated than for unsigned. For $\alpha \in B_N$, it can be shown that
\begin{equation}\label{g0}
\alpha_{N-1} = 0 \quad \Rightarrow \quad \alpha \ge 0
\end{equation}
\begin{equation}\label{l0}
\alpha_{N-1} = 1 \quad \Rightarrow \quad \alpha < 0
\end{equation}

For both, the maximum and minimum are obtained by setting 
subsequent bits to $1$ and $0$, respectively. 

Therefore, for $\alpha \in a, \beta \in b$, and $i < N-1$,
\begin{equation} \label{sand0}
(\alpha \& \beta)_{N-1} = 0 \quad \Rightarrow \quad c_{max, N-1} = 0
\end{equation}
\begin{equation}
(\alpha\&\beta)_i = 1 \quad \Rightarrow \quad 
c_{max,i} = 1
\end{equation}

Note this corresponds to $unsigned \, max\{a\&b\}$.

The possibilities for $a_{min,N-1}$ and $a_{max,N-1}$ are 
slightly different 
than for $a^i_{min,i}$ and $a^i_{max,i}$:
\begin{equation}\label{11s} 
a_{min,N-1},a_{max,N-1} = 1,1
\end{equation}
\begin{equation}\label{10s} 
a_{min,N-1},a_{max,N-1} = 1,0
\end{equation}
\begin{equation}\label{00s} 
a_{min,N-1}, a_{max,N-1} = 0,0
\end{equation}

Now we have
\begin{equation}\label{sup1}
a_{max,N-1} = 0 \quad \Rightarrow \quad \alpha_i = 0 \quad 
\text{for some} \; \alpha \in a
\end{equation}
\begin{equation}\label{sdn1}
a_{min,N-1} = 1 \quad \Rightarrow \quad \alpha_i = 1 \quad 
\text{for some} \; \alpha \in a
\end{equation}

Then by \eqref{sand0}
\begin{equation*}
c_{max,N-1} = a_{max,N-1} \& b_{max,N-1}
\end{equation*}

For signed integers at bit $N-1$, $up$ and $dn$ may not have the desired 
effect. Note that now $\vec 0$ is the smallest integer with bit $N-1$ set 
to $0$, and $\vec 1$ is the largest integer with bit $N-1$ set to $1$.

These definitions maintain $up(a)$ is the smallest signed integer with bit 
$N-1$ set to $0$, and $dn(a)$ is the largest signed integer with bit 
$N-1$ set to $1$.

For $\eqref{10s} \wedge b_{min,N-1},b_{max,N-1} = 00$, $\hat a_{N-1}$ is unconstrained,
but $a_i = 1$ will help to maximize $c_{max}$, so

\begin{equation*}
a^{N-2} = [a_{min}, \vec 1]
\end{equation*}

For $\eqref{10s} \wedge b_{min,N-1},b_{max,N-1} = 11$, it must be that $\hat a_{N-1} = 0$,
therefore

\begin{equation*}
a^{N-2} = [\vec 0, a_{max}]
\end{equation*}

For $\eqref{10s} \wedge b_{min,N-1},b_{max,N-1} = 10$, then we need at least
one of $\hat a_{N-1} = 0$, $\hat b_{N-1} = 0$. However, we would also like
$a_{max,i} = 1$ and $b_{max,i} = 1$ for to maximize $c_{max}$. 
Therefore, suppose we choose $\hat a_{N-1} = 1$. Then $a^{N-2}_{max} = \vec 1$.
Note this has the effect that $s(c_{max},N-2) = s(b_{max},N-2)$. 
Obviously, this only maximizes $c_{max}$ when $s(a_{max},N-2) \le 
s(b_{max},N-2)$.

Therefore, when $a_{max} \ge b_{max}$
\begin{equation*}
a^{N-2} = [\vec 0, a_{max}]
\end{equation*}
\begin{equation*}
b^{N-2} = [b_{min}, \vec 1]
\end{equation*}

When $a_{max} < b_{max}$
\begin{equation*}
a^{N-2} = [a_{min}, \vec 1]
\end{equation*}
\begin{equation*}
b^{N-2} = [\vec 0, b_{max}]
\end{equation*}

Then the algorithm for unsigned may be used, however we may need to constrain
the high bit to satisfy \eqref{sand0}. Thus: 
\begin{lstlisting}
int32_t maxand_s(int32_t amin, int32_t amax, int32_t bmin, int32_t bmax){
    int32_t a0 = amin & ~amax &  bmax & 0x80000000;
    int32_t a1 = amin & ~amax & ~bmin & 0x80000000;
    int32_t b0 = bmin & ~bmax &  amax & 0x80000000;
    int32_t b1 = bmin & ~bmax & ~amin & 0x80000000;
    int32_t ab = amin & ~amax & bmin & ~bmax & 0x80000000;
    if( a1 || (ab && amax <  bmax)) amax = -1;
    if( a0 || (ab && amax >= bmax)) amin = 0;
    if( b1 || (ab && amax >= bmax)) bmax = -1;
    if( b0 || (ab && amax <  bmax)) bmin = 0;
    return maxand(amin, amax, bmin, bmax);
}
\end{lstlisting}

\section{signed max\{a\text{\textbar}b\}}

For $\alpha \in a, \beta \in b$, and $i < N-1$,

\begin{equation} 
(\alpha \text{\textbar} \beta)_{N-1} = 0 \quad \Rightarrow \quad 
c_{max, N-1} = 0
\end{equation}
\begin{equation}
(\alpha\text{\textbar}\beta)_i = 1 
\quad \Rightarrow \quad 
c_{max,i} = 1
\end{equation}

Then

\begin{equation*}
c_{max,N-1} = a_{max,N-1} \text{\textbar} b_{max,N-1}
\end{equation*}

For $\eqref{10s} \wedge b_{min,N-1},b_{max,N-1} = 10$, we must have 
$\hat a_{N-1} = 0$ and likewise for $b$ for $c_{max,N-1} = 0$. 

For $\eqref{10s} \wedge b_{min,N-1},b_{max,N-1} = 00$, then 
$\hat a_{N-1} = 0$ to maximize $c_{max}$.

For $\eqref{10s} \wedge b_{min,N-1},b_{max,N-1} = 11$, then $\hat a_{N-1}$ 
is unconstrained, however $c_{max}$ is maximized when $\hat a_i = 1$. 

So for $\eqref{10s} \wedge b_{max,N-1} = 0$, 
\begin{equation*}
a^{N-2} = [\vec 0, a_{max}]
\end{equation*}

For $\eqref{10s} \wedge b_{max,N-1} = 1$, 
\begin{equation*}
a^{N-2} = [a_{min},\vec 1]
\end{equation*}

\begin{lstlisting}
int32_t maxor_s(int32_t amin, int32_t amax, int32_t bmin, int32_t bmax){
    int32_t a1 = (amin & ~amax &  bmax) & 0x80000000;
    int32_t a0 = (amin & ~amax & ~bmax) & 0x80000000;
    int32_t b1 = (bmin & ~bmax &  amax) & 0x80000000;
    int32_t b0 = (bmin & ~bmax & ~amax) & 0x80000000;
    if( a1 ) amax = -1;
    if( a0 ) amin = 0;
    if( b1 ) bmax = -1;
    if( b0 ) bmin = 0;
    return maxor(amin, amax, bmin, bmax);
}
\end{lstlisting}

\section{signed min \{a\&b\}}
For $\alpha \in a, \beta \in b$, and $i < N-1$
\begin{equation} 
(\alpha \& \beta)_{N-1} = 1 \quad \Rightarrow \quad c_{min, N-1} = 1
\end{equation}
\begin{equation}\label{maxand+}
(\alpha\&\beta)_i = 0 \quad \Rightarrow \quad 
c_{min,i} = 0
\end{equation}

By \eqref{10s},
\begin{equation*} 
(a_{min} \& b_{min})_{N-1} = 1 \quad \Rightarrow \quad c_{min, N-1} = 1
\end{equation*}

Then
\begin{equation*} 
c_{min,N-1} = (a_{min} \& b_{min})_{N-1} 
\end{equation*}

As with unsigned, $[\sim a_{max}, \sim a_{min}]$ is a valid range. Also,
\begin{equation*}
\sim (\sim a_{min} \text{\textbar} \sim b_{min}) = (a_{min} \& b_{min})
\end{equation*}

Then we can do the same as for unsigned:
\begin{lstlisting}
int32_t minand_s(int32_t amin, int32_t amax, int32_t bmin, int32_t bmax){
    return ~maxor_s(~amax, ~amin, ~bmax, ~bmin);
}
\end{lstlisting}

\section{signed min\{a\text{\textbar}b\}}

Likewise,
\begin{lstlisting}
int32_t minor_s(int32_t amin, int32_t amax, int32_t bmin, int32_t bmax){
    return ~maxand_s(~amax, ~amin, ~bmax, ~bmin);
}
\end{lstlisting}

\section{signed $max\{a$\^{}$b\}$}

For $\alpha \in a, \beta \in b$, and $i < N-1$,

\begin{equation} 
(\alpha \text{\^{}} \beta)_{N-1} = 0 \quad \Rightarrow \quad 
c_{max, N-1} = 0
\end{equation}
\begin{equation}
(\alpha\text{\^{}}\beta)_i = 1 
\quad \Rightarrow \quad 
c_{max,i} = 1
\end{equation}

Decomposed:
\begin{equation*}
a_{min,N-1} = 1 \wedge b_{min,N-1} = 1 \vee 
a_{max,N-1} = 0 \wedge b_{max,N-1} = 0
\quad \Rightarrow \quad 
c_{max,N-1} = 0
\end{equation*}

Then:
\begin{equation*}
c_{max,N-1} = (a_{min,N-1}\text{\^{}}b_{min,N-1})\&(a_{max,N-1}\text{\^{}}b_{max,N-1})
\end{equation*}

For $\eqref{10s} \wedge b_{min,N-1},b_{max,N-1} = 00$, then it must be that 
$\hat a_{N-1} = 0$. Therefore

\begin{equation*}
a^{N-2} = [\vec 0, a_{max}]
\end{equation*}

For $\eqref{10s} \wedge b_{min,N-1},b_{max,N-1} = 11$, then it must be that
$\hat a_{N-1} = 1$. Therefore

\begin{equation*}
a^{N-2} = [a_{min},\vec 1]
\end{equation*}

For $\eqref{10s}\wedge b_{min,N-1},b_{max,N-1} = 10$, then we need either
both set to $1$ or both set to $0$, but which? Meh, screw it. Try each,
and take the maximum of the two results.

So:
\begin{equation*}
a^{N-2} = [a_{min},\vec 1]
\end{equation*}
\begin{equation*}
b^{N-2} = [b_{min},\vec 1]
\end{equation*}

or
\begin{equation*}
a^{N-2} = [\vec 0,a_{max}]
\end{equation*}
\begin{equation*}
b^{N-2} = [\vec 0,b_{max}]
\end{equation*}

\begin{lstlisting}
int32_t maxxor_s(int32_t amin, int32_t amax, int32_t bmin, int32_t bmax){
    int32_t a0 = amin & ~amax & ~bmin & 0x80000000;
    int32_t a1 = amin & ~amax &  bmax & 0x80000000;
    int32_t b0 = bmin & ~bmax & ~amin & 0x80000000;
    int32_t b1 = bmin & ~bmax &  amax & 0x80000000;
    int32_t ab = amin & ~amax & bmin & ~bmax & 0x80000000;
    if( ab ){
        return max( maxxor(0,amax,0,bmax), maxxor(amin,-1,bmin,-1));
    }
    if( a0 ) amin = 0;
    if( a1 ) amax = -1;
    if( b0 ) bmin = 0;
    if( b1 ) bmax = -1;
    return maxxor(amin,amax,bmin,bmax);
}
\end{lstlisting}

\section{signed $min\{a$\^{}$b\}$}

For $\alpha \in a, \beta \in b$, and $i < N-1$,

\begin{equation} 
(\alpha \text{\^{}} \beta)_{N-1} = 1 \quad \Rightarrow \quad 
c_{min, N-1} = 1
\end{equation}
\begin{equation}
(\alpha\text{\^{}}\beta)_i = 0 
\quad \Rightarrow \quad 
c_{min,i} = 0
\end{equation}

Then:
\begin{equation*}
c_{min,N-1} = (a_{min,N-1}\text{\^{}}b_{max,N-1})\text{\textbar}(a_{max,N-1}\text{\^{}}b_{min,N-1})
\end{equation*}

For $\eqref{10s} \wedge b_{min,N-1},b_{max,N-1} = 00$, then it must be that 
$\hat a_{N-1} = 1$. Therefore

\begin{equation*}
a^{N-2} = [a_{min},\vec 1]
\end{equation*}

For $\eqref{10s} \wedge b_{min,N-1},b_{max,N-1} = 11$, then it must be that
$\hat a_{N-1} = 0$. Therefore

\begin{equation*}
a^{N-2} = [\vec 0, a_{max}]
\end{equation*}

For $\eqref{10s}\wedge b_{min,N-1},b_{max,N-1} = 10$, then both are 
unconstrained. Then we want $\hat a_{N-1} = 0$ and $\hat b_{N-1} = 1$,
or vice versa - but which? Meh, screw it, try each and take the minimum of 
the two results.

So:
\begin{equation*}
a^{N-2} = [\vec 0, a_{max}]
\end{equation*}
\begin{equation*}
b^{N-2} = [b_{min},\vec 1]
\end{equation*}

or
\begin{equation*}
a^{N-2} = [a_{min},\vec 1]
\end{equation*}
\begin{equation*}
b^{N-2} = [\vec 0, b_{max}]
\end{equation*}

\begin{lstlisting}
int32_t minxor_s(int32_t amin, int32_t amax, int32_t bmin, int32_t bmax){
    int32_t a0 = amin & ~amax &  bmax & 0x80000000;
    int32_t a1 = amin & ~amax & ~bmin & 0x80000000;
    int32_t b0 = bmin & ~bmax &  amax & 0x80000000;
    int32_t b1 = bmin & ~bmax & ~amin & 0x80000000;
    int32_t ab = amin & ~amax & bmin & ~bmax & 0x80000000;
    if( ab ) {
        return min( minxor(0, amax, bmin, -1), minxor(amin, -1, 0, bmax));
    }
    if( a0 ) amin = 0;
    if( a1 ) amax = -1;
    if( b0 ) bmin = 0;
    if( b1 ) bmax = -1;
    return minxor(amin,amax,bmin,bmax);
}
\end{lstlisting}
\end{document}
